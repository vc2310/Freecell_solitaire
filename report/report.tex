\documentclass[12pt]{article}

\usepackage{graphicx}
\usepackage{paralist}
\usepackage{amsfonts}
\usepackage{amsmath}
\usepackage{hhline}
\usepackage{booktabs}
\usepackage{multirow}
\usepackage{multicol}

\oddsidemargin 0mm
\evensidemargin 0mm
\textwidth 160mm
\textheight 200mm
\renewcommand\baselinestretch{1.0}

\pagestyle {plain}
\pagenumbering{arabic}

\newcounter{stepnum}

%% Comments

\usepackage{color}

\newif\ifcomments\commentstrue

\ifcomments
\newcommand{\authornote}[3]{\textcolor{#1}{[#3 ---#2]}}
\newcommand{\todo}[1]{\textcolor{red}{[TODO: #1]}}
\else
\newcommand{\authornote}[3]{}
\newcommand{\todo}[1]{}
\fi

\newcommand{\wss}[1]{\authornote{blue}{SS}{#1}}

\title{Assignment 4, Part 1, Specification}
\author{SFWR ENG 2AA4}

\begin {document}

\maketitle

Submit your design specification, written in LaTeX, of the MIS for the game
state module.  If your specification requires additional modules, you should
include their MISes as well.  It is up to you to determine your modules
interface; that is, you decide on the exported constants, access programs,
exceptions etc.  You also determine your state variables and specify the
semantics for your access program calls.  Your design does not need to concern
itself with performance.


\newpage


\section* {Cards ADT Module}

\subsection*{Template Module}

CardsT

\subsection* {Uses}

N/A

\subsection* {Syntax}

\subsubsection* {Exported Types}

CardsT = ?

\subsubsection* {Exported Access Programs}

\begin{tabular}{| l | l | l | l |}
\hline
\textbf{Routine name} & \textbf{In} & \textbf{Out} & \textbf{Exceptions}\\
\hline
CardsT & Strings, Strings & CardsT & \\
\hline
getRank & ~ & Strings & ~\\
\hline
getSuit & ~ & Strings & ~\\
\hline
getColour & ~ & Strings & ~\\
\hline
\end{tabular}

\subsection* {Semantics}

\subsubsection* {State Variables}

$ranks$: Strings\\
$suits$: Strings

\subsubsection* {State Invariant}

None

\subsubsection* {Assumptions}

The constructor CardsT is called for each object instance before any other
access routine is called for that object.  The constructor cannot be called on
an existing object.

\subsubsection* {Access Routine Semantics}

CardsT($r, s$):
\begin{itemize}
\item transition: $ranks, suits := r, s$
\item output: $out := \mathit{self}$
\item exception: None
\end{itemize}

\noindent getRank():
\begin{itemize}
\item output: $out := ranks$
\item exception: None
\end{itemize}

\noindent getSuit():
\begin{itemize}
\item output: $out := suits$
\item exception: None
\end{itemize}

\noindent getColour():
\begin{itemize}
	\item output: $out := "Heart" \lor "Diamond" \implies "Red" | "Black"$
	\item exception: None
\end{itemize}

\newpage

\section* {CardsGame Module}

\subsection*{Template Module}

Game

\subsection* {Uses}

CardsT

\subsection* {Syntax}

\subsubsection* {Exported Types}

CardsDeck = ?

\subsubsection* {Exported Access Programs}

\begin{tabular}{| l | l | l | l |}
	\hline
	\textbf{Routine name} & \textbf{In} & \textbf{Out} & \textbf{Exceptions}\\
	\hline
	CardsDeck &  & CardsDeck & \\
	\hline
	ShufflingDeck & ~ & ~ & ~\\
	\hline
	DealingCards & ~ & ~ & ~\\
	\hline
	moveCard & $\mathit{Z},\mathit{Z}$ & Sequence of (Sequence of CardsT) & ~\\
	\hline
	moveCardFreecell & $\mathit{Z}$ & ~ & ~\\
	\hline
	moveCardFoundation & $\mathit{Z},\mathit{Z}$ & ~ & ~\\
	\hline
	checkValidMove & $\mathit{Z},\mathit{Z}$ & $\mathit{B}$ & ~\\
	\hline
	checkValidMoveFreecell & ~ & $\mathit{B}$ & ~\\
	\hline
	checkValidMoveFoundation & $\mathit{Z},\mathit{Z}$ & $\mathit{B}$ & ~\\
	\hline
	winningSituation & ~ & $\mathit{B}$ & ~\\
	\hline
	
\end{tabular}

\subsection* {Semantics}

\subsubsection* {State Variables}

$deck[52]$: CardsT\\
$columns$: Sequence of (Sequence of CardsT)\\
$freecell$: Sequence of CardsT\\
$foundation$: Sequence of (Sequence of CardsT)\\

\subsubsection* {State Invariant}

SIZE = 52

\subsubsection* {Assumptions}

The constructor CardsDeck is called for each object instance before any other
access routine is called for that object.  The constructor cannot be called on
an existing object.

\subsubsection* {Access Routine Semantics}

CardsDeck():
\begin{itemize}
	\item transition: $i\in\mathit{N}|i\in[0..SIZE]:deck[i]=CardsT(ranks[i \% 13], suits[i / 13])$
	\item output: $out := \mathit{self}$
	\item exception: None
\end{itemize}

\noindent ShufflingDeck():
\begin{itemize}
	\item transition: $i\in\mathit{N},j\in\mathit{N}|i\in[0..SIZE],j = (rand() + time(0)) \% SIZE:deck[i],deck[j]=deck[j],deck[i]$
	\item exception: None
\end{itemize}

\noindent DealingCards():
\begin{itemize}
	\item transition: $i\in\mathit{N}|i\in[0..7]:i=0\implies deck[0..6]|i=1\implies deck[7..13]|i=2\implies deck[14..20]|i=3\implies deck[21..27]|i=4\implies deck[28..33]|i=5\implies deck[34..39]|i=6\implies deck[40..45]|i=7\implies deck[46..51]|$
	\item exception: None
\end{itemize}

\noindent moveCard(from,to):
\begin{itemize}
	\item transition: $(checkValidMove(from,to) = true) \implies (columns.at(to).back() = columns.at(from).back())$
	\item output: $out :=columns$
	\item exception: None
\end{itemize}

\noindent moveCardFreecell(from):
\begin{itemize}
	\item transition: $(checkValidMoveFreecell() = true) \implies (freecell.back() = columns.at(from).back())$
	\item exception: None
\end{itemize}

\noindent moveCardFoundation(from,to):
\begin{itemize}
	\item transition: $(checkValidMoveFoundation(from,to) = true) \implies (foundation.at(to).back() = columns.at(from).back())$
	\item output: $out :=columns$
	\item exception: None
\end{itemize}

\noindent checkValidMove(from,to):
\begin{itemize}
	\item output: $out := (columns.at(to).back().getRank().index() - columns.at(from).back().getRank().index() = 1) \land (columns.at(from).back().getColour() != columns.at(to).back().getColour()) \implies true$
	\item exception: None
\end{itemize}

\noindent checkValidMoveFreecell():
\begin{itemize}
	\item output: $out :=freecell.size() < 4 \implies true$
	\item exception: None
\end{itemize}

\noindent checkValidMoveFoundation(from,to):
\begin{itemize}
	\item output: $out := (foundation.at(to).back().getRank().index() - columns.at(from).back().getRank().index() = 1) \land (columns.at(from).back().getSuit() = foundation.at(to).back().getSuit()) \implies true$
	\item exception: None
\end{itemize}

\noindent winningSituation():
\begin{itemize}
	\item output: $+(i\in\mathit{N}|i\in[0..4]|foundation.at(i).size() = 13 : 1) = 4 \implies true$
	\item exception: None
\end{itemize}

\end {document}